\documentclass{article}
\textwidth = 420pt
\oddsidemargin = 15pt

\usepackage{color}
\usepackage{hyperref}
\usepackage{amssymb, amsmath, amsfonts, amsthm}
\usepackage{thmtools}
\usepackage{algorithm} 
\usepackage{enumitem}

\newcommand{\mc}{\mathcal}
\setlength{\parindent}{24pt}
\renewcommand{\bar}[1]{\overline{#1}} 

\newtheorem{theorem}{Theorem}
\newtheorem{fact}[theorem]{Fact}
\newtheorem{conj}[theorem]{Conjecture}
\newtheorem{problem}[theorem]{Problem}
\newtheorem{definition}[theorem]{Definition}
\newtheorem{lemma}[theorem]{Lemma}
\newtheorem{proposition}[theorem]{Proposition}
\newtheorem{corollary}[theorem]{Corollary}
\newtheorem{example}[theorem]{Example}

\newcommand{\ConjName}[1]{\label{con:#1}}
\newcommand{\Conj}[1]{Conjecture~\ref{con:#1}}
\newcommand{\ProblemName}[1]{\label{prob:#1}}
\newcommand{\Problem}[1]{Problem~\ref{prob:#1}}
\renewcommand{\dot}{\bullet}
\newcommand{\Tr}{\operatorname{tr}}
\newcommand{\eps}{\epsilon}

\newcommand{\lmax}{\lambda_\mathrm{max}}
\newcommand{\lmin}{\lambda_\mathrm{min}}
\newcommand{\ufinal}{u_\mathrm{final}}
\newcommand{\lfinal}{l_\mathrm{final}}
\newcommand{\umax}{u_\mathrm{max}}

\newcommand{\Symraw}{\mathbb{S}}
\newcommand{\Sym}[1][]{\Symraw^{\ifthenelse{\equal{#1}{}}{m}{#1}}}
\newcommand{\Psd}[1][]{\Symraw_+^{\ifthenelse{\equal{#1}{}}{m}{#1}}}
\newcommand{\Reals}{\mathbb{R}}
\newcommand{\iprod}[2]{\langle #1, #2 \rangle}
\newcommand{\paren}[2][]{#1({#2}#1)}
\newcommand{\qform}[2]{\transp{#2}#1#2}
\newcommand{\transp}[1]{#1^T}



% Simple (outer) environment for algorithms
\newenvironment{outer_alg}{
    \begin{list}{}{
        \setlength{\itemsep}{2pt}
        \setlength{\parsep}{0pt}
        \setlength{\parskip}{0pt}
        \setlength{\topsep}{1pt}
        \setlength{\leftmargin}{5pt}
    }
}
{
    \end{list}
}

% Simple environments for algorithms
\newenvironment{alg}{
    \begin{list}{}{
        \setlength{\itemsep}{2pt}
        \setlength{\parsep}{0pt}
        \setlength{\parskip}{0pt}
        \setlength{\topsep}{1pt}
    }
}
{
    \end{list}
}

\DeclareMathOperator{\argmin}{argmin}

%%%%% Title %%%%%
\title{\LARGE OneDigit Schema}
\author{}


%%%%% Document Body %%%%%
\begin{document}
\maketitle

\section{Notation}
Let $\mathcal{A}$ be the set of alphabet. We assume that $|\mathcal{A}|=N$. For the case of passowrd generation, $\mathcal{A}=\{A,B,\ldots,Z\}$ and $N=26$. We denote the set of digits by $\mathcal{D}$, i.e., $\mathcal{D}=\{0,\ldots,9\}$. Let's $\mathcal{C}$ denotes the set of possible challenges. For the sake of simplicity, we assume that each challenge $c\in\mathcal{C}$ does not contain more than four repeated letters. We denote the $i^{th}$ coordinate of a vector $\vec{u}$ by $u_i$.

\section{OneDigit Schema}

%%%%%%%%%%%%%%%%%%%%%%%%%%%%%%%%%%%%%%%%%%%%%%%%%%%%%%%%%%%%%%%%%%%%%%%%%%%
\subsection{Preprocessing step}
\begin{itemize}
	\item[$\cdot$] Memorize a a random map $f:\mathcal{A} \to \mathcal{D}$
	\item[$\cdot$] Memorize a random string $s = s_1 \ldots s_{d-1}\in \mathcal{D}^{d-1}$
\end{itemize}

\subsection{Processing step}

\begin{algorithm}
\label{OneDigit}
\begin{alg}
\item[] Input: Challenge $c=c_1 \ldots c_l$
\item[] $g \overset{10}{\equiv} f(c_1)+\ldots +f(c_l)$
\item[] Output: Response $sg$
\label{alg:NotKnown}
\end{alg}
\caption{OneDigit schema}
\end{algorithm}

Before stating the main theorem of this note, we define the notion of strong linearly independence.

\begin{definition}
We say that set of challenges $\{c_1,\ldots,c_p\}$ is strong linearly independent$\pmod{10}$ if $\{c_1,\ldots,c_p\}$ is linearly independent$\pmod{5}$ and$\pmod{2}$. Note that a direct consequence of strong linear indepence is linear independence.
\end{definition}
                                                                                                            

\begin{theorem}
\label{thm:main}
	Denote the output of OneDigit schema on a challenge $c$, by $p(c)$. We define $\mathcal{R}=\{p(c) \, | \, c\in\mathcal{C}\}$. For any challenge $c\in \mathcal{C}$ and any response $r\in \mathcal{R}$
	\begin{itemize} 
		\item[$(a)$]$\Pr[p(c)=r]=\dfrac{1}{10^d}$
	\end{itemize}
Furthermore, assume that we have made $k$ observations $\big(c_1,p(c_1)\big),\ldots, \big(c_k, p(c_k)\big)$. Then, $\forall g_{k+1}\in\mathcal{D}$ and $\forall c_{k+1}\in \mathcal{C}$ s.t. $\{c_1,\ldots,c_k,c_{k+1}\}$ is strong linearly independent$\pmod{10}$ 
	\begin{itemize}
		\item[$(b)$] $\Pr[p(c_{k+1})=s g_{k+1} \, | \, (p(c_1)=s g_1),\ldots, (p(c_k)=s g_k)]=1/10$
	\end{itemize}
Part $(a)$ is saying that without having any prior information, the probability of guessing the correct response to any single challenge is ${1}/{10^d}$. In other words, for any two responses $r_1$ and $r_2$
$$\Pr[p(c)=r_1]=\Pr[p(c)=r_2]$$

\noindent Now assume that the adversary has observed $k$ (input, output) pairs and she is trying to guess the response to a new challenge $c_{k+1}$. After seeing the first (input, output) pair, she will know the value of $s$. So the only unknown part of $p(c_{k+1})$ is the single digit $g_{k+1}$. Part $(b)$ is saying that for any new challenge $c_{k+1}$ which forms a strong linearly independent set with $k$ observed challenges, the adversary can't do better than guessing $g_{k+1}$ randomly.

\end{theorem}


\begin{proof}

\begin{itemize}
	\item[(a)]

 For any $c\in\mathcal{C}, r\in\mathcal{R}$. Let $r=r_1\ldots r_d$
	$$
	\begin{array}{lcl}
		\Pr[p(c)=r] & = &
		\Pr[p(c)_1\ldots p(c)_{d-1}=r_1\ldots r_{d-1}]\Pr[p(c)_d=r_d] \\
		& =&
	 \Pr[s=r_1\ldots r_{d-1}]\Pr[p(c)_d=r_d]
	\end{array}
	$$
	Since each digit of string $s$ is chosen independently at random, the above formula is equal to
	$$
	\Pr[s_1=r_1]\ldots \Pr[s_{d-1}=r_{d-1}]\Pr[r(c)_d=r_d]
	$$
	 The first $d-1$ probabilities appearing above are each equal to $1/10$. Thus we only need to compute $\Pr[r(c)_d=r_d]= 
	        \Pr[f(c_1)+\ldots+f(c_{l})\equiv r_d \pmod{10} ]$. One way to compute this probability is to count the number of maps $f$ that satisfy 
	\begin{equation}
		\label{eq:probability}
	        f(c_1)+\ldots+f(c_{l})\equiv r_d \pmod{10}	
	    \end{equation}

	  and divide it by the total number of maps $f:\mathcal{A}\to \mathcal{D}$. What is the number of maps $f$ that satisfy Eq.~\ref{eq:probability} ? One can choose $f(c_1),\ldots ,f(c_{l-1})$ arbitrarily, then $f(c_l)$ will be chosen uniquely by $f(c_l)\equiv r_d-\sum_{i=1}^{l-1}f(c_i)\pmod{10}$. So the total number of choices of $f$ will be $10^{N-{l}}$ for the letters that are not present in $c$, $10^{{l-1}}$ for the first $l-1$ letters in $c$ and 1 for the last letter in $c$. So the total number of choices is $10^{N-l}10^{l-1}=10^{N-1}$. Note that the total number of maps $f:\mathcal{A}\to \mathcal{D}$ is $10^{N}$. This leads to
	$$
	\Pr[r(c)_d=r_d]=\Pr[f(c_1)+\ldots+f(c_l)=r_d]=\dfrac{10^{N-1}}{10^{N}}=\frac{1}{10}
	$$
	Consequently, accounting for the fixed string $s$
	$$
	\Pr[r(c)=r]=\frac{1}{10^{d-1}}\frac{1}{10}=\frac{1}{10^d}
	$$
	
\item[(b)]
	Now assume that the adversary have observed $k$ $($challenge, response$)$ pairs 
	$\big(c_1, p(c_1)=s g_1\big)$,\\$\ldots$,$\big(c_k, p(c_k)=s g_k\big)$, and we want to compute
	$$
	\Pr[\big(p(c_{k+1})=s g_{k+1}\big) \, | \, \big(p(c_1)=s g_1\big),\ldots, \big(p(c_k)=s g_k\big)]
	$$
	This is equal to
	$$
	 \Pr[\big(p(c_{k+1})_d=g_{k+1}\big)\, | \, \big(p(c_1)_d=g_1\big),\ldots, \big(p(c_k)_d=g_k\big)]
	 $$
	 which is equal to 
	 \begin{equation}
	 \label{eq:probability}
	 \frac{\Pr[\big(p(c_{k+1})_d=g_{k+1}\big), \big(p(c_1)_d=g_1\big),\ldots, \big(p(c_k)_d=g_k\big)]}{\Pr[\big(p(c_1)_d=g_1\big),\ldots, \big(p(c_k)_d=g_k\big)}	
	 \end{equation}
	 
%	$$
%	\Pr[r(c_{k+1})=s^*d \, | \, (c_1,s^*d_1),\ldots, (c_1,s^*d_k)]=
%	 \dfrac{\Pr[(c_1,s^*d_1),\ldots, (c_1,s^*d_k), (c_{k+1},s^*d)]}{\Pr[(c_1,s^*d_1),\ldots, (c_1,s^*d_k)]}
%	$$
	We start by computing the value of denominator. The nominator value can be achieved similarly. In order to compute $\Pr[\big(p(c_1)_d=g_1\big),\ldots, \big(p(c_k)_d=g_k\big)]$, we should count the number of mappings $f$ that satisfy 
	
	\begin{equation}
	\label{linearSet}
	\left\{ \begin{array}{l}
     f(c_{11})+\ldots+ f(c_{1l})\equiv g_1 \pmod{10}\\
     \vdots       \\
     f(c_{k1})+\ldots+ f(c_{kl})\equiv g_{k} \pmod{10}
    \end{array} \right.	
	\end{equation}
In the next lemma, we show that the number of solutions to above $k$ linear equations is $10^{N-{k}}$. Therefore, the value of the ratio~\eqref{eq:probability} is equal to
$$
\frac{10^{n-{k+1}}}{10^{n-k}} = \frac{1}{10}
$$
\end{itemize}
\end{proof}

\begin{lemma}
\label{lem:linearIndep}
	Given a function $f:\mathcal{A}\to \mathcal{D}$ and set $\{c_1,\ldots,c_k\} \subseteq \mathcal{C}$ strong linearly independent and $g_1,\ldots,g_{k+1}\in\mathcal{D}$, the system of linear equations~\ref{linearSet}, has $10^{N-{k}}$ solutions. 
\end{lemma}	
\begin{proof}
Assume there is an ordering $a_1,\ldots,a_N$ on elements of $\mathcal{A}$. Let's define the $N$-dimensional column vector $\vec{f}$ such that $\vec{f}_i=f(a_i)$ s.t. $a_i$ is the $i^{th}$ element of $\mathcal{A}$. Similarly, for every challenge $c\in\mathcal{C}$, we define the $N$-dimensional row vector $\vec{c}$ as follows. The $i^{th}$ coordinate of $\vec{c}$, $\vec{c}_i$, is the number of occurrence of the $a_i$ in $c$. In this vector setting, the last system of equations will be equivalent to 


	\begin{equation}
	\label{eq:matrix}
	\left\{ \begin{array}{lcr}
     \vec{c_1}\cdot \vec{f} \overset{10}{\equiv} g_1 &  \\
     \vdots  & \Rightarrow &  
     
     \left[ \begin{array}{l}
     \vec{c}_1 \\
     \vdots \\
     \vec{c}_{k} \\
     \end{array}
     \right]     \cdot\vec{f}\overset{10}{\equiv}
     
     \left[ \begin{array}{l}
     {g}_1 \\
     \vdots \\
     {g}_{k} \\
     \end{array}
     \right]     
      \\
     
     \vec{c}_{k}\cdot \vec{f} \overset{10}{\equiv} g_{k} & &
    \end{array} \right.
    \end{equation}
   	Let's define
   	$$
     C= \left[ \begin{array}{l}
     \vec{c}_1 \\
     \vdots \\
     \vec{c}_{k}
	 \end{array}\right] 
	 ,\,\,\, \vec{g} = 
	 \left[
	 \begin{array}{l}
	 	g_1\\
	    \vdots \\
	 	g_k
	 \end{array}
	 \right] 
   	$$
   	
By assumption, rows of matrix $C$ are linearly independent$\pmod{2}$, thus there must be $k$ columns $\{C^{j_1},\ldots,C^{j_k}\}$ that are linearly independent$\pmod{2}$. Using Prop.~\ref{prop:mod2}, $\{C^{j_1},\ldots,C^{j_k}\}$ are linearly independent$\pmod{5}$ as well.
   
   
   We claim that for any set $\mathcal{F}_{N-k}=\{\vec{f}_j\in\mathcal{D}\, : \, j \in \{j_1,\ldots,j_k\}^c \}$, there will be a unique set $\mathcal{F}_k=\{\vec{f}_j\in\mathcal{D}:\,\, j\in\{j_1,\ldots,j_k\}\}$ such that $\mathcal{F}_k \cup\mathcal{F}_{N-k}$ is a solution for system \ref{eq:matrix}.
 
%   This will lead to $10^{26-{k+1}}=10^{25-k}$ different choices for $\vec{f}$.

 
  Given a set $\mathcal{F}_{N-k}$, let's substitute arbitrary values of $f_j$ for $j\notin\{j_1,\ldots,j_{k}\}$ in Eq.~3. So the matrix equation will be simplified as follows
 
 $$
  \begin{array}{l}
 [C^{j_1},\ldots, C^{j_k}] 
 \left[ \begin{array}{l}
 f_{j_1} \\
 \vdots \\
 f_{j_k}
 \end{array}	 \right] \overset{10}{\equiv}
 \left[\begin{array}{l}
 g'_{j_1} \\
 \vdots \\
 g'_{j_k}
 \end{array}\right] 
  \end{array} 
  $$
 
Therefore 
\begin{equation}
\label{eq:mod5}
  \begin{array}{l}
 [C^{j_1},\ldots, C^{j_k}] 
 \left[ \begin{array}{l}
 f_{j_1} \\
 \vdots \\
 f_{j_k}
 \end{array}	 \right] \overset{5}{\equiv}
 \left[\begin{array}{l}
 g'_{j_1} \\
 \vdots \\
 g'_{j_k}
 \end{array}\right],
 
  \begin{array}{l}
 [C^{j_1},\ldots, C^{j_{k}}] 
 \left[ \begin{array}{l}
 f_{j_1} \\
 \vdots \\
 f_{j_{k}}
 \end{array}	 \right] \overset{2}{\equiv}
 \left[\begin{array}{l}
 g'_{j_1} \\
 \vdots \\
 g'_{j_k}
 \end{array}\right] 
 \end{array} 
 
 
 \end{array} 
  \end{equation}
 
Since matrix $[C^{j_1},\ldots,C^{j_k}]$ is full rank$\pmod{5}$ and$\pmod{2}$, the above linear equations respectively has unique solution $\vec{x}\pmod{5}$ and $\vec{y}\pmod{2}$. We claim that, there exists a vector $\vec{z}$ such that $\vec{z}\equiv\vec{x}\pmod{5}$ and $\vec{z}\equiv\vec{y}\pmod{2}$. To prove our claim, we first need to briefly remind Chinese Remainder Theorem.

\begin{theorem}(Chinese Remainder Theorem)
\label{chinese}
Suppose $n_1, ..., n_k$ are positive integers that are pairwise coprime. Then, for any given sequence of integers $a_1, ..., a_k$, there exists an integer $x$ solving the following system of simultaneous congruences.
$$
\left\{ \begin{array}{cl}
     x \equiv a_1 &\pmod{n_1} \\
      \vdots &\\
      x \equiv a_k &\pmod{n_k} \\
\end{array} \right.		
$$
Furthermore, any two solutions of this system are congruent modulo the product $N = n_1 \ldots n_k$. Hence, there is a unique (non-negative) solution less than $N$.
\end{theorem}

We want to prove that there exists a vector $\vec{z}$ such that $\vec{z}\equiv\vec{x}\pmod{5}$ and $\vec{z}\equiv\vec{y}\pmod{2}$. Consider the following $k$ systems of simultaneous congruences:
$$
\begin{array}{lcr}
\begin{array}{l}
{z}_1\overset{5}{\equiv} {x}_1\\
{z}_1\overset{2}{\equiv} {y}_1
\end{array} &
\begin{array}{c}
\ldots
\end{array}&
\begin{array}{r}
{z}_k\overset{5}{\equiv} {x}_k\\
{z}_k\overset{2}{\equiv} {y}_k
\end{array}
	
\end{array}
$$
Using Chinese Remainder Theorem, there exist ${z}_1,\ldots,{z}_k$ satisfying the above congruences. Furthermore, for all $i\in[k]$, ${z}_k\pmod{10}$ is unique. Therefore $z=[z_1,\ldots,z_k]\transp{}$ will be the unique solution to Eq.~\ref{eq:mod5}. 


So far we have shown that every set $\mathcal{F}_{N-k}=\{\vec{f}_j\in\mathcal{D}\, : \, j \notin \{j_1,\ldots,j_k\} \}$, there is a unique set $\mathcal{F}_k=\{\vec{f}_j\in\mathcal{D}:\,\, j\in\{j_1,\ldots,j_k\}\}$ such that $\mathcal{F}_k \cup\mathcal{F}_{N-k}$ is a solution for system \ref{eq:matrix}. Therefore, the number of solutions to system of linear equations \ref{eq:matrix} is number of sets $\mathcal{F}_{N-k}=\{\vec{f}_j\in\mathcal{D}\, : \, j \notin \{j_1,\ldots,j_k\} \}$ which is equal to $10^{N-k}$.
\end{proof}

\begin{proposition}
\label{prop:mod2}
Given a set of challenges $\mathcal{C}=\{c_1,\ldots,c_k\}$ s.t. $\forall i\in[k]$, the challenge $c_i$ does not contain more than four repeated letters. If $\mathcal{C}$ is linearly independent$\pmod{2}$, this implies that it is linearly independent$\pmod{5}$ as well.
\end{proposition}
\begin{proof}
	
\end{proof}
	
	






%%%%%%%%%%%%%%%%%%%%%%%%%%%%%%%%%%%%%%%%%%%%%%%%%%%%%%%%%%%%%%%%%%%%%%%%%%%
\subsection{HUM}
In order to calculate the HUM, we first need to write the steps of Alg.~\ref{alg:NotKnown} in more details

\begin{algorithm}
\begin{alg}
\item[] Input: Challenge $c=c_1 \ldots c_l$
\item[] Set $i=1$, SUM$=0$
\item[] While not EndOfChallenge :

\begin{itemize}
	\item[] $\begin{array}{lr}
		\text{Compute } f(c_i) & \text{(Applying the map)} \\
		\text{SUM} \overset{10}{\equiv} \text{SUM} + f(c_i) & \text{(Add to the running sum)} \\
			i=i+1 & \text{(shift pointer)} 
	\end{array}$
\end{itemize}

\item[] Print fixed string $s$
\item[] Print SUM

\label{alg:HUM}
\end{alg}
\caption{OneDigit schema}
\end{algorithm}

HUM =  $2 \times$ (initialization) + $l \times$ (while loop condition) + $l \times$ (map) + $l \times$ (add) + $l \times$ (shift pointer) + (end while) + $2 \times$ (print) = $4l+5$


\subsection{Security: $Q$ value}

Theorem.~\ref{thm:main} is saying that as long as a new challenge forms a strong linearly independent set with already observed challenges, the adversary can not predict the response to this new challenge. In order to compute the value of $Q$ for OneDigit schema we should answer the following question:
\begin{itemize}
	\item[$\diamond$] Given a set of challenges $\mathcal{C}$, at each round, a new challenge $c \in \mathcal{C}$ is chosen uniformly at random. Let $\mathcal{C}_i$ be the set of challenges chosen till round $i$. What is the maximum $i$
	such that $\mathcal{C}_i$ is a strong linearly independent set?
	\end{itemize}
To answer the above question, we ran the following experiment. We chose the challenge set $\mathcal{C}$ to be the set of all the valid website names. At each iteration, our program choses a challenge $c$ uniformly at random from $\mathcal{C}$ and checks if $c$ along with challenges chosen so far, forms a strong linearly independent set. If yes, it saves the number of iterations as the $Q$ value. Otherwise, it will continue. We repeated this procedure for $1000$ times and took the average of all the saved $Q$ values. The following table shows the result:

\vspace{3mm}


\begin{center}
\begin{tabular}{c|r}
Number of trials              & Q  \\
\hline
1000              &   $\sim 18$             \\
\end{tabular}
\label{QValue}
\end{center}





\end{document}
 