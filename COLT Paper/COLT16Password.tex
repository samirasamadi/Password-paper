\documentclass[anon,12pt]{colt2016}
% Anonymized submission

\title[Learning Humanly Usable and Secure Password Schemas]{Learning Humanly Usable and Secure Password Schemas}
\usepackage{times} 
%\usepackage{epsfig} % less modern
\usepackage{enumitem}
\usepackage{bm}
\usepackage{xcolor}
\usepackage{tikz}



\coltauthor{\Name{Manuel Blum} \Email{mblum@cs.cmu.edu }\\
 \addr Carnegie Mellon University
 \AND
 \Name{Santosh S Vempala} \Email{vempala@cc.gatech.edu }\\
 \addr Georgia Institute of Technology
 \AND
 \Name{Samira Samadi} \Email{ssamadi6@gatech.edu}\\
 \addr Georgia Institute of Technology
 }


% For citations
\usepackage{natbib}

% For algorithms
\usepackage{algorithm}
\usepackage{algorithmic}

\usepackage{nickstyle_ICML}

\usepackage{hyperref}

\usepackage{color}
\usepackage{hyperref}
\usepackage[toc,page]{appendix}
\usepackage{xspace}

\renewcommand{\th}{\ifmmode{^{\textrm{th}}}\else{\textsuperscript{th}\ }\fi}

\newcommand\overmat[2]{%
  \makebox[0pt][l]{$\smash{\color{white}\overbrace{\phantom{%
    \begin{matrix}#2\end{matrix}}}^{\text{\color{black}#1}}}$}#2}
\newcommand\bovermat[2]{%
  \makebox[0pt][l]{$\smash{\overbrace{\phantom{%
    \begin{matrix}#2\end{matrix}}}^{\text{#1}}}$}#2}

%%%%% Macros %%%%%
\setlength{\parindent}{24pt}
\renewcommand{\bar}[1]{\overline{#1}}

\newtheorem{fact}[theorem]{Fact}
\newtheorem{conj}[theorem]{Conjecture}
\newtheorem{problem}[theorem]{Problem}
\newtheorem{question}[theorem]{Question}
\newcommand{\ConjName}[1]{\label{con:#1}}
\newcommand{\Conj}[1]{Conjecture~\ref{con:#1}}
\newcommand{\ProblemName}[1]{\label{prob:#1}}
\newcommand{\Problem}[1]{Problem~\ref{prob:#1}}
\renewcommand{\dot}{\bullet}
\newcommand{\Tr}{\operatorname{tr}}
\newcommand{\eps}{\epsilon}
\newcommand{\smallnorm}[1]{\lVert #1 \rVert}

\newcommand{\lmax}{\lambda_\mathrm{max}}
\newcommand{\lmin}{\lambda_\mathrm{min}}
\newcommand{\ufinal}{u_\mathrm{final}}
\newcommand{\lfinal}{l_\mathrm{final}}
\newcommand{\umax}{u_\mathrm{max}}

\newcommand{\Symraw}{\mathbb{S}}
\newcommand{\Sym}[1][]{\Symraw^{\ifthenelse{\equal{#1}{}}{m}{#1}}}
\newcommand{\Psd}[1][]{\Symraw_+^{\ifthenelse{\equal{#1}{}}{m}{#1}}}
\newcommand{\Reals}{\mathbb{R}}
\newcommand{\iprod}[2]{\langle #1, #2 \rangle}
\newcommand{\paren}[2][]{#1({#2}#1)}
\newcommand{\qform}[2]{\transp{#2}#1#2}
\newcommand{\transp}[1]{#1^T}
\newcommand{\convhull}{\operatorname{conv.hull}}
\newcommand{\poly}{\operatorname{poly}}

\newcommand{\sv}{\mathbf{sv}}
\renewcommand{\ss}{\mathbf{ss}}
\newcommand{\st}{\mathbf{st}}

\newcommand{\PP}{Permutation Problem\xspace}
\newcommand{\HP}{Sampling Problem\xspace}
\newcommand{\HA}{Herding algorithm\xspace}

\usepackage{float}
\usepackage{caption2}


% Simple environments for algorithms
\newenvironment{alg}{
    \begin{list}{}{
        \setlength{\itemsep}{2pt}
        \setlength{\parsep}{0pt}
        \setlength{\parskip}{0pt}
        \setlength{\topsep}{1pt}
    }
}
{
    \end{list}
}

\floatstyle{ruled}
 \newfloat{algorithm}{ht}{lop}
\floatname{algorithm}{Algorithm}
\makeatletter
\renewcommand{\floatc@ruled}[2]{\vspace{3pt}{\@fs@cfont #1:\:} #2 \par \vspace{2pt}}
\makeatother


\begin{document} 

\maketitle


\begin{abstract} 
	
\end{abstract}

\begin{keywords}
List of keywords
\end{keywords}



%%%%%%%%%%%%%%%%%%%%%%%%%%%%%%%%%%%%%%%%%%%%%%%%%%%%%%%%%%%%%%%%%%%%%%%%%%%%%%%%%%%%%%
\section{Introduction}
\SectionName{intro}

%%%%%%%%%%%%%%%%%%%%%%%%%%%%%%%%%%%%%%%%%%%%%%%%%%%%%%%%%%%%%%%%%%%%%%%%%%%%%%%%%%%%%%
\subsection{Human Usability of Password Schemas}
\SectionName{HUMschema}

%%%%%%%%%%%%%%%%%%%%%%%%%%%%%%%%%%%%%%%%%%%%%%%%%%%%%%%%%%%%%%%%%%%%%%%%%%%%%%%%%%%%%%
\subsection{Security Measures for Password Schemas}
\SectionName{secSchema}

%%%%%%%%%%%%%%%%%%%%%%%%%%%%%%%%%%%%%%%%%%%%%%%%%%%%%%%%%%%%%%%%%%%%%%%%%%%%%%%%%%%%%%
\section{Digit Schema}
\SectionName{digitSchema}

\subsection{Notation}
\SectionName{notation}

Let $\mathcal{A}$ be the set of alphabet. We assume that $|\mathcal{A}|=N$. For the case of passowrd generation, $\mathcal{A}=\{A,B,\ldots,Z\}$ and $N=26$. We denote the set of digits by $\mathcal{D}$, i.e., $\mathcal{D}=\{0,\ldots,9\}$. Let's $\mathcal{C}$ denotes the set of possible challenges. We denote the $i^{th}$ coordinate of a vector $\vec{u}$ by $u_i$. Given a positive integer $k$, we define $[k]=\{0,\ldots,k-1\}$

%%%%%%%%%%%%%%%%%%%%%%%%%%%%%%%%%%%%%%%%%%%%%%%%%%%%%%%%%%%%%%%%%%%%%%%%%%%%%%%%%%%%%%%%%%%%%%%%%%%%%%%%%%%%%%%%%%%%%%%%%%%%%
\subsection{Description}
\SectionName{description}

\noindent \textbf{Preprocessing step}

\begin{itemize}
	\item[$\cdot$] Memorize a a random map $f:\mathcal{A} \to \mathcal{D}$
	\item[$\cdot$] Memorize a random string $s = s_1 \ldots s_{d-1}\in \mathcal{D}^{d-1}$
\end{itemize}


\noindent \textbf{Processing Step}

\begin{algorithm}
\label{OneDigit}
\begin{alg}
\item[] Input: Challenge $c=c_1 \ldots c_l$
\item[] $g \overset{10}{\equiv} f(c_1)+\ldots +f(c_k)$
\item[] Output: Response $sg$
\label{alg:NotKnown}
\end{alg}
\caption{Digit schema}
\end{algorithm}

\begin{itemize}
	\item \textcolor{red}{[S	AMIRA]:} Should we write the Digit schema in the general format (reading k characters of challenge) here? i think it's better to keep the simple algorithm here and also do the analysis on this algorithm. In the later sections [maybe experiments], we can say that: As the experiments show, even reading the first few letters of the challenge is enough to get high values of $Q$, so one can only compute the parity of say the first three letters.
\end{itemize}
%%%%%%%%%%%%%%%%%%%%%%%%%%%%%%%%%%%%%%%%%%%%%%%%%%%%%%%%%%%%%%%%%%%%%%%%%%%%%%%%%%%%%%

\subsection{Human Usability Model}
\SectionName{HUM}


In order to calculate the HUM, we first need to write the steps of Alg.~\ref{alg:NotKnown} in more details

\begin{algorithm}
\begin{alg}
\item[] Input: Challenge $c=c_1 \ldots c_l$
\item[] Set $i=1$, SUM$=0$
\item[] While not EndOfChallenge :

\begin{itemize}
	\item[] $\begin{array}{lr}
		\text{Compute } f(c_i) & \text{(Applying the map)} \\
		\text{SUM} \overset{10}{\equiv} \text{SUM} + f(c_i) & \text{(Add to the running sum)} \\
			i=i+1 & \text{(shift pointer)} 
	\end{array}$
\end{itemize}

\item[] Print fixed string $s$
\item[] Print SUM

\label{alg:HUM}
\end{alg}
\caption{Digit schema}
\end{algorithm}

HUM =  $2 \times$ (initialization) + $l \times$ (while loop condition) + $l \times$ (map) + $l \times$ (add) + $l \times$ (shift pointer) + (end while) + $2 \times$ (print) = $4l+5$


%%%%%%%%%%%%%%%%%%%%%%%%%%%%%%%%%%%%%%%%%%%%%%%%%%%%%%%%%%%%%%%%%%%%%%%%%%%%%%%%%%%%%%
\section{Security Analysis}
\SectionName{secAnalysis}

%%%%%%%%%%%%%%%%%%%%%%%%%%%%%%%%%%%%%%%%%%%%%%%%%%%%%%%%%%%%%%%%%%%%%%%%%%%%%%%%%%%%%%
\subsection{Best Possible $Q$}
\SectionName{optimalQ}

\begin{itemize}
	\item \textcolor{red}{[SAMIRA]:}TODO
\end{itemize}

%%%%%%%%%%%%%%%%%%%%%%%%%%%%%%%%%%%%%%%%%%%%%%%%%%%%%%%%%%%%%%%%%%%%%%%%%%%%%%%%%%%%%%
\subsection{Digit Schema achieves best possible $Q$ for strong linearly independent challenges}
\SectionName{linIndep}

We assume that challenges are coming uniformly at random from the set $\mathcal{C}$. Lemma~\ref{lem:kIndep} shows that with high probability the first $N$ challenges will be strong linearly independent$\pmod{10}$. Therefore, the probability that the adversary guess the correct response for the $N^{th}$ challenge is still $1/10$.

\begin{definition}
Given prime numbers $p$ and $q$, we say that set of challenges $\{c_1,\ldots,c_n\}$ is strong linearly independent$\pmod{pq}$, if $\{c_1,\ldots,c_n\}$ is linearly independent$\pmod{p}$ and$\pmod{q}$. Note that a direct consequence of strong linear indepence is linear independence.
\end{definition}
                            
%%%%%%%%%%%%%%%%%%%%%%%%%%%%%%%%%%%%%%%%%%%%%%%%%%%%%%%%%%%%%%%%%%%%%%%%%%%%%%%%%%%                                                                                                           
\begin{theorem}
\label{thm:main}
	Denote the output of OneDigit schema on a challenge $c$, by $p(c)$. We define $\mathcal{R}=\{p(c) \, | \, c\in\mathcal{C}\}$. For any challenge $c\in \mathcal{C}$ and any response $r\in \mathcal{R}$
	\begin{itemize} 
		\item[$(a)$]$\Pr[p(c)=r]=\dfrac{1}{10^d}$
	\end{itemize}
Furthermore, assume that we have made $k$ observations $\big(c_1,p(c_1)\big),\ldots, \big(c_k, p(c_k)\big)$. Then, $\forall g_{k+1}\in\mathcal{D}$ and $\forall c_{k+1}\in \mathcal{C}$ s.t. $\{c_1,\ldots,c_k,c_{k+1}\}$ is strong linearly independent$\pmod{10}$ 
	\begin{itemize}
		\item[$(b)$] $\Pr[p(c_{k+1})=s g_{k+1} \, | \, (p(c_1)=s g_1),\ldots, (p(c_k)=s g_k)]=1/10$
	\end{itemize}
\end{theorem}

Part $(a)$ is saying that without having any prior information, the probability of guessing the correct response to any single challenge is ${1}/{10^d}$. In other words, for any two responses $r_1$ and $r_2$
$$\Pr[p(c)=r_1]=\Pr[p(c)=r_2]$$

\noindent Now assume that the adversary has observed $k$ (challenge, response) pairs and she is trying to guess the response to a new challenge $c_{k+1}$. After seeing the first (challenge, response) pair, she will know the value of $s$. So the only unknown part of $p(c_{k+1})$ is the single digit $g_{k+1}$. Part $(b)$ is saying that for any new challenge $c_{k+1}$ which forms a strong linearly independent set with $k$ previously observed challenges, the adversary can not do better than guessing $g_{k+1}$ completely randomly.

Lemma~\ref{lem:linearIndepp} and lemma~\ref{lem:indeppq} provide the main tools in proving Theorem~\ref{thm:main}.




%%%%%%%%%%%%%%%%%%%%%%%%%%%%%%%%%%%%%%%%%%%%%%%%%%%%%%%%%%%%%%%%%%%%%%%%%%%%%%%%%%%%%%

\begin{lemma}
\label{lem:linearIndepp}
Given a prime number $p$, matrix $C_{k\times N}$ with $k$ linearly independent rows$\pmod{p}$, and column vector ${g}$, the number of solutions to $C{x} \overset{p}{\equiv} {g}$ is $p^{N-k}$.
\end{lemma}


\begin{proof}  	
By assumption, rows of matrix $C$ are linearly independent$\pmod{p}$, thus there must be $k$ columns $\{C^{j_1},\ldots,C^{j_k}\}$ that are linearly independent$\pmod{p}$.   
   Let's denote the $j^{th}$ coordinate of vector $x$ by $x_j$. We claim that for any set $\mathcal{X}_{N-k}=\{{x}_j\in\{0,\ldots,p-1\}\, : \, j \notin \{j_1,\ldots,j_k\} \}$, there will be a unique set $\mathcal{X}_k=\{{x}_j\in\{0,\ldots,p-1\}:\,\, j\in\{j_1,\ldots,j_k\}\}$ such that $x=\mathcal{X}_k \cup\mathcal{X}_{N-k}$ is a solution for system $C{x}\overset{p}{\equiv} {g}$.
 
  Given an arbitrary set $\mathcal{X}_{N-k}$, let's substitute values of $x_j$ for $j\notin\{j_1,\ldots,j_{k}\}$ in ${x}$ and simplify the equation $Cx\overset{p}{\equiv} g$ :
 
 $$
  \begin{array}{l}
 [C^{j_1},\ldots, C^{j_k}] 
 \left[ \begin{array}{l}
 x_{j_1} \\
 \vdots \\
 x_{j_k}
 \end{array}	 \right] \overset{p}{\equiv}
 \left[\begin{array}{l}
 g'_{1} \\
 \vdots \\
 g'_{k}
 \end{array}\right] 
  \end{array} 
  $$
Where $g'_{l}= g_l -\sum_{i\notin \{j_1,\ldots,j_k\}} C_{li}x_{i}$ for all $l\in\{1,\ldots,k\}$. 

Since matrix $[C^{j_1},\ldots,C^{j_k}]$ is full rank$\pmod{p}$, the above linear equation has unique solution ${x}\pmod{p}$. So far we have proved that for every set $\mathcal{X}_{N-k}=\{{x}_j\in\{0,\ldots,p-1\}\, : \, j \notin \{j_1,\ldots,j_k\} \}$, there is a unique set $\mathcal{X}_k=\{{x}_j\in\{0,\ldots,p-1\}:\,\, j\in\{j_1,\ldots,j_k\}\}$ such that $x=\mathcal{X}_k \cup\mathcal{X}_{N-k}$ is a solution for system $C{x}\overset{p}{\equiv} {g}$. This concludes that the number of solutions to $C{x}\overset{p}{\equiv} {g}$ is equal to the number of possible sets $\mathcal{X}_{N-k}$ which is equal to $p^{N-k}$. 

\end{proof}
%%%%%%%%%%%%%%%%%%%%%%%%%%%%%%%%%%%%%%%%%%%%%%%%%%%%%%%%%%%%%%%%%%%%%%%%%%%%%%%%%%%

\begin{lemma}
\label{lem:indeppq}
Given prime numbers $p$ and $q$, matrix $C_{k\times N}$ with $k$ strong linearly independent rows $\pmod{pq}$, and column vector ${g}$, the number of solutions to $C{x} \overset{pq}{\equiv} {g}$ is $(pq)^{N-k}$.
\end{lemma}

\begin{proof}
	

We want to count the number of solutions to $C{x} \overset{pq}{\equiv}{g}$. Let's denote $S_p$ to be the set of solutions to $Cx\overset{p}{\equiv}g$ and similarly $S_q$ to be the set of solutions to $Cx\overset{q}{\equiv}g$. Using lemma~\ref{lem:linearIndepp}, $|S_p|=p^{N-k}$ and $|S_q|=q^{N-k}$. We claim that for every $x\in S_p$ and $y \in S_q$ there is a unique vector $z\pmod{pq}$ such that $z$ is a solution for $Cz\overset{pq}{\equiv}g$. Furthermore, for every such vector $z\pmod{pq}$, there is a unique pair $x\in S_p$ and $y\in S_q$. 
 
Given $x\in S_p$, and $y\in S_q$, we prove that there exists a $k$-dimensional vector ${z}$ such that ${z}\equiv{x}\pmod{p}$ and ${z}\equiv{y}\pmod{q}$. Consider the following $k$ systems of simultaneous congruences:
$$
\begin{array}{lcr}
\begin{array}{l}
{z}_1\overset{p}{\equiv} {x}_1\\
{z}_1\overset{q}{\equiv} {y}_1
\end{array} &
\begin{array}{c}
\ldots
\end{array}&
\begin{array}{r}
{z}_k\overset{p}{\equiv} {x}_k\\
{z}_k\overset{q}{\equiv} {y}_k
\end{array}
	
\end{array}
$$
Using Chinese Remainder Theorem, there exist ${z}_1,\ldots,{z}_k$ satisfying the above congruences. Therefore
$$
\begin{array}{lrr}
z \overset{p}{\equiv} x  & \Rightarrow Cz \overset{p}{\equiv} Cx\overset{p}{\equiv} g &\\
& & \Rightarrow Cz \overset{pq}{\equiv} g \\

z \overset{q}{\equiv} y  &\Rightarrow Cz \overset{q}{\equiv} Cx\overset{q}{\equiv} g &
\end{array}
$$
 Furthermore, for all $i\in[k]$, ${z}_i\pmod{pq}$ is unique. Therefore $z=[z_1,\ldots,z_k]\transp{}$ will be the unique solution to $Cx\overset{pq}{\equiv}g$. So far we have shown that for every pair $(x,y)\in S_p\times S_q$ there is one and only one vector $z\pmod{pq}$ satisfying $Cz\overset{pq}{\equiv}g$. Therefore, the number of solutions to $Cz\overset{pq}{\equiv}$ is equal to $|S_p||S_q|=(pq)^{N-k}$
 
\end{proof}



%%%%%%%%%%%%%%%%%%%%%%%%%%%%%%%%%%%%%%%%%%%%%%%%%%%%%%%%%%%%%%%%%%%%%%%%%%%%%%%%%%%

\begin{lemma}(Chinese Remainder Theorem)
\label{chinese}
Suppose $n_1, ..., n_k$ are positive integers that are pairwise coprime. Then, for any given sequence of integers $a_1, ..., a_k$, there exists an integer $x$ solving the following system of simultaneous congruences.
$$
\left\{ \begin{array}{cl}
     x \equiv a_1 &\pmod{n_1} \\
      \vdots &\\
      x \equiv a_k &\pmod{n_k} \\
\end{array} \right.		
$$
Furthermore, any two solutions of this system are congruent modulo the product $N = n_1 \ldots n_k$. Hence, there is a unique (non-negative) solution less than $N$.

\end{lemma}


%%%%%%%%%%%%%%%%%%%%%%%%%%%%%%%%%%%%%%%%%%%%%%%%%%%%%%%%%%%%%%%%%%%%%%%%%%%%%%%%%%%
Now we have all the requirements to prove Theorem~\ref{thm:main}.
\begin{proof}(Theorem~\ref{thm:main})


\begin{itemize}
	\item[(a)]

 For any $c\in\mathcal{C}, r\in\mathcal{R}$. Let $r=r_1\ldots r_d$
	$$
	\begin{array}{lcl}
		\Pr[p(c)=r] & = &
		\Pr[p(c)_1\ldots p(c)_{d-1}=r_1\ldots r_{d-1}]\Pr[p(c)_d=r_d] \\
		& =&
	 \Pr[s=r_1\ldots r_{d-1}]\Pr[p(c)_d=r_d]
	\end{array}
	$$
	Since each digit of string $s$ is chosen independently at random, the above formula is equal to
	$$
	\Pr[s_1=r_1]\ldots \Pr[s_{d-1}=r_{d-1}]\Pr[r(c)_d=r_d]
	$$
	 The first $d-1$ probabilities appearing above are each equal to $1/10$. Thus we only need to compute $\Pr[r(c)_d=r_d]= 
	        \Pr[f(c_1)+\ldots+f(c_{l})\equiv r_d \pmod{10} ]$. One way to compute this probability is to count the number of maps $f$ that satisfy 
	\begin{equation}
		\label{eq:probability}
	        f(c_1)+\ldots+f(c_{l})\equiv r_d \pmod{10}	
	    \end{equation}

	  and divide it by the total number of maps $f:\mathcal{A}\to \mathcal{D}$. What is the number of maps $f$ that satisfy Eq.~\ref{eq:probability} ? One can choose $f(c_1),\ldots ,f(c_{l-1})$ arbitrarily, then $f(c_l)$ will be chosen uniquely by $f(c_l)\equiv r_d-\sum_{i=1}^{l-1}f(c_i)\pmod{10}$. So the total number of choices of $f$ will be $10^{N-{l}}$ for the letters that are not present in $c$, $10^{{l-1}}$ for the first $l-1$ letters in $c$ and 1 for the last letter in $c$. So the total number of choices is $10^{N-l}10^{l-1}=10^{N-1}$. Note that the total number of maps $f:\mathcal{A}\to \mathcal{D}$ is $10^{N}$. This leads to
	$$
	\Pr[r(c)_d=r_d]=\Pr[f(c_1)+\ldots+f(c_l)=r_d]=\dfrac{10^{N-1}}{10^{N}}=\frac{1}{10}
	$$
	Consequently, accounting for the fixed string $s$
	$$
	\Pr[r(c)=r]=\frac{1}{10^{d-1}}\frac{1}{10}=\frac{1}{10^d}
	$$
	
\item[(b)]
	Now assume that the adversary have observed $k$ $($challenge, response$)$ pairs 
	$\big(c_1, p(c_1)=s g_1\big),\ldots,\big(c_k, p(c_k)=s g_k\big)$, and we want to compute
	$$
	\Pr[\big(p(c_{k+1})=s g_{k+1}\big) \, | \, \big(p(c_1)=s g_1\big),\ldots, \big(p(c_k)=s g_k\big)]
	$$
	This is equal to
	$$
	 \Pr[\big(p(c_{k+1})_d=g_{k+1}\big)\, | \, \big(p(c_1)_d=g_1\big),\ldots, \big(p(c_k)_d=g_k\big)]
	 $$
	 which is equal to 
	 \begin{equation}
	 \label{eq:probability}
	 \frac{\Pr[\big(p(c_{k+1})_d=g_{k+1}\big), \big(p(c_1)_d=g_1\big),\ldots, \big(p(c_k)_d=g_k\big)]}{\Pr[\big(p(c_1)_d=g_1\big),\ldots, \big(p(c_k)_d=g_k\big)}	
	 \end{equation}
	 
%	$$
%	\Pr[r(c_{k+1})=s^*d \, | \, (c_1,s^*d_1),\ldots, (c_1,s^*d_k)]=
%	 \dfrac{\Pr[(c_1,s^*d_1),\ldots, (c_1,s^*d_k), (c_{k+1},s^*d)]}{\Pr[(c_1,s^*d_1),\ldots, (c_1,s^*d_k)]}
%	$$
	We start by computing the value of denominator. The nominator value can be achieved similarly. In order to compute $\Pr[\big(p(c_1)_d=g_1\big),\ldots, \big(p(c_k)_d=g_k\big)]$, we should count the number of mappings $f$ that satisfy 
	
	\begin{equation}
	\label{linearSet}
	\left\{ \begin{array}{l}
     f(c_{11})+\ldots+ f(c_{1l})\equiv g_1 \pmod{10}\\
     \vdots       \\
     f(c_{k1})+\ldots+ f(c_{kl})\equiv g_{k} \pmod{10}
    \end{array} \right.	
	\end{equation}
Lemma~\ref{lem:indeppq} shows that the number of solutions to above $k$ linear equations is $10^{N-{k}}$. Therefore, the value of the ratio~\eqref{eq:probability} is equal to
$$
\frac{10^{n-{k+1}}}{10^{n-k}} = \frac{1}{10}
$$
\end{itemize}

	
\end{proof}





%%%%%%%%%%%%%%%%%%%%%%%%%%%%%%%%%%%%%%%%%%%%%%%%%%%%%%%%%%%%%%%%%%%%%%%%%%%%%%%%%%%

\begin{lemma}

	$Q = N$ for lin. ind. challenges.
	\begin{itemize}
		\item \textcolor{red}{SAMIRA]:} TODO
	\end{itemize}
\end{lemma}



%%%%%%%%%%%%%%%%%%%%%%%%%%%%%%%%%%%%%%%%%%%%%%%%%%%%%%%%%%%%%%%%%%%%%%%%%%%%%%%%%%%
\begin{lemma}
\label{lem:kIndep}
Given prime numbers $p$ and $q$ and matrix $C_{k\times N}$, such that each row $C_i$ is chosen uniformly at random from $[pq]^{N}$. With probability higher than ... rows of $C$ are strong linearly independent$\pmod{pq}$.
	
\end{lemma}
\begin{proof}
	What are the number of $k\times N$ matrices which has $k$ linearly independent rows$\pmod{p}$? There are $p^{N}-1$ choices for the first row, as we can chose anything except the all zeros vector. We want the second row to be linearly independent$\pmod{p}$ of the first row. Therefore, we can chose any $C_2\notin\{0,C_1,2C_1,\ldots,(p-1)C_1\}$. This eliminates $p$ choices and therefore the number of choices for the second row will $p^{N}-p$. Similarly, the $k^{th}$ row should not belong to any linear combination of already chosen $k-1$ rows. Hence the number of choices for the $k^{th}$ row will be $p^N-p^{k-1}$. Therefore, the number of matrices of size $k\times N$ and rank $k\pmod{p}$ is $\Pi_{i=0}^{k-1}(p^N-p^i)$. Similarly, the number of matrices of size $k\times N$ and rank $k\pmod{q}$ is $\Pi_{i=0}^{k-1}(q^N-q^i)$. 
	
	Given a matrix $X_{k\times N}$ with rank $k\pmod{p}$ and matrix $Y_{k\times N}$ with rank $k\pmod{q}$, we define matrix $Z_{k\times N}$ as following.
	$$
	Z_i \equiv qX_i + pY_i\pmod{pq} 
	$$
It is easy to see that rows of $Z$ are linearly independent both$\pmod{p}$ and$\pmod{q}$ and hence strong linearly independent$\pmod{pq}$. Also note that the above transfomation $(X,Y)\to Z $ is one-to-one i.e., for every two pairs of matrices $(X,Y)$ and $(X',Y')$,  
$$
\text{if }
\left\{ \begin{array}{l}
X_i \neq X'_i\pmod{p}\\
\text{or} \\
Y_i \neq Y'_i\pmod{q}
\end{array}\right.
   \Rightarrow qX_i+pY_i \neq qX'_i+pY'_i\pmod{pq}
$$  
Therefore, the number of matrices of size $k\times N$ with $k$ strong linearly independent rows$\pmod{pq}$ is at least $\Pi_{i=0}^{k-1}(p^N-p^i)\Pi_{i=0}^{k-1}(q^N-q^i)$
Dividing it by the total number of $k\times N$ matrices$\pmod{pq}$ we get that the probability that a random matrix of size $k\times N$ has $k$ strong linearly independent rows$\pmod{pq}$ is bigger than
$$
\frac{\Pi_{i=0}^{k-1}(p^N-p^i)\Pi_{i=0}^{k-1}(q^N-q^i)}{(pq)^{kN}}
$$
\begin{itemize}
	\item {\textcolor{red}{[TASK]} Find a concrete lower bound for the above probability}
\end{itemize}
\end{proof}

%%%%%%%%%%%%%%%%%%%%%%%%%%%%%%%%%%%%%%%%%%%%%%%%%%%%%%%%%%%%%%%%%%%%%%%%%%%%%%%%%%%%%%

\subsection{Experimental evaluation of $Q$ for websites, English words and random strings.}
\SectionName{experiment}

Theorem.~\ref{thm:main} is saying that as long as a new challenge forms a strong linearly independent set with already observed challenges, the adversary can not predict the response to this new challenge. In order to compute the value of $Q$ for OneDigit schema we should answer the following question:
\begin{itemize}
	\item[$\diamond$] Given a set of challenges $\mathcal{C}$, at each round, a new challenge $c \in \mathcal{C}$ is chosen uniformly at random. Let $\mathcal{C}_i$ be the set of challenges chosen till round $i$. What is the maximum $i$
	such that $\mathcal{C}_i$ is a strong linearly independent set?
	\end{itemize}
To answer the above question, we ran the following experiment. We chose the challenge set $\mathcal{C}$ to be the set of all the valid website names. At each iteration, our program choses a challenge $c$ uniformly at random from $\mathcal{C}$ and checks if $c$ along with challenges chosen so far, forms a strong linearly independent set. If yes, it saves the number of iterations as the $Q$ value. Otherwise, it will continue. We repeated this procedure for $1000$ times and took the average of all the saved $Q$ values. The following table shows the result:

\vspace{3mm}


\begin{center}
\begin{tabular}{c|r}
Number of trials              & Q  \\
\hline
1000              &   $\sim 18$             \\
\end{tabular}
\label{QValue}
\end{center}




%%%%%%%%%%%%%%%%%%%%%%%%%%%%%%%%%%%%%%%%%%%%%%%%%%%%%%%%%%%%%%%%%%%%%%%%%%%%%%%%%%%%%%

\section{Conclusion and Questions}
\SectionName{conc}
%%%%%%%%%%%%%%%%%%%%%%%%%%%%%%%%%%%%%%%%%%%%%%%%%%%%%%%%%%%%%%%%%%%%%%%%%%%%%%%%%%%%%%




\clearpage
\bibliography{COLT}



%%%%%%%%%%%%%%%%%%%%%%%%%%%%%%%%%%%%%%%%%%%%%%%%%%%%%%%%%%%%%%%%%%%%%%%%%%%%%%%%%%%%%%
\appendix
%\section{}
%\label{app:appendixA}

\end{document}
